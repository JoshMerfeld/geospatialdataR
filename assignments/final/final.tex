% Options for packages loaded elsewhere
\PassOptionsToPackage{unicode}{hyperref}
\PassOptionsToPackage{hyphens}{url}
%
\documentclass[
]{article}
\usepackage{amsmath,amssymb}
\usepackage{iftex}
\ifPDFTeX
  \usepackage[T1]{fontenc}
  \usepackage[utf8]{inputenc}
  \usepackage{textcomp} % provide euro and other symbols
\else % if luatex or xetex
  \usepackage{unicode-math} % this also loads fontspec
  \defaultfontfeatures{Scale=MatchLowercase}
  \defaultfontfeatures[\rmfamily]{Ligatures=TeX,Scale=1}
\fi
\usepackage{lmodern}
\ifPDFTeX\else
  % xetex/luatex font selection
\fi
% Use upquote if available, for straight quotes in verbatim environments
\IfFileExists{upquote.sty}{\usepackage{upquote}}{}
\IfFileExists{microtype.sty}{% use microtype if available
  \usepackage[]{microtype}
  \UseMicrotypeSet[protrusion]{basicmath} % disable protrusion for tt fonts
}{}
\makeatletter
\@ifundefined{KOMAClassName}{% if non-KOMA class
  \IfFileExists{parskip.sty}{%
    \usepackage{parskip}
  }{% else
    \setlength{\parindent}{0pt}
    \setlength{\parskip}{6pt plus 2pt minus 1pt}}
}{% if KOMA class
  \KOMAoptions{parskip=half}}
\makeatother
\usepackage{xcolor}
\usepackage[margin=1in]{geometry}
\usepackage{longtable,booktabs,array}
\usepackage{calc} % for calculating minipage widths
% Correct order of tables after \paragraph or \subparagraph
\usepackage{etoolbox}
\makeatletter
\patchcmd\longtable{\par}{\if@noskipsec\mbox{}\fi\par}{}{}
\makeatother
% Allow footnotes in longtable head/foot
\IfFileExists{footnotehyper.sty}{\usepackage{footnotehyper}}{\usepackage{footnote}}
\makesavenoteenv{longtable}
\usepackage{graphicx}
\makeatletter
\def\maxwidth{\ifdim\Gin@nat@width>\linewidth\linewidth\else\Gin@nat@width\fi}
\def\maxheight{\ifdim\Gin@nat@height>\textheight\textheight\else\Gin@nat@height\fi}
\makeatother
% Scale images if necessary, so that they will not overflow the page
% margins by default, and it is still possible to overwrite the defaults
% using explicit options in \includegraphics[width, height, ...]{}
\setkeys{Gin}{width=\maxwidth,height=\maxheight,keepaspectratio}
% Set default figure placement to htbp
\makeatletter
\def\fps@figure{htbp}
\makeatother
\setlength{\emergencystretch}{3em} % prevent overfull lines
\providecommand{\tightlist}{%
  \setlength{\itemsep}{0pt}\setlength{\parskip}{0pt}}
\setcounter{secnumdepth}{5}
\usecolortheme{beaver}
\useinnertheme{rounded}
\useoutertheme[subsection=false,footline=authortitle]{miniframes}  

\beamertemplatenavigationsymbolsempty
\setbeamertemplate{headline}{}

\setbeamerfont{block title}{size={}}

\definecolor{kdisgreen}{RGB}{0, 99, 52}
\definecolor{kdisplatinum}{RGB}{167, 169, 172}

\setbeamercolor{alerted text}{fg=kdisgreen}
\setbeamercolor{example text}{fg=kdisplatinum}

\setbeamercolor*{palette secondary}{fg=white,bg=kdisplatinum} %subsection
\setbeamercolor*{palette tertiary}{fg=white,bg=kdisgreen} %section

\setbeamercolor{title}{fg=black}         %Title of presentation
\setbeamercolor{section title}{fg=white, bg=kdisgreen}
\setbeamercolor{subsection title}{fg=kdisgreen, bg=kdisplatinum}
\setbeamercolor{frametitle}{fg=kdisgreen, bg=kdisplatinum}

\setbeamerfont{frametitle}{size=\small}
\setbeamerfont{frametitle}{series=\bfseries}

\setbeamertemplate{blocks}[rounded][shadow=true] 
\setbeamertemplate{items}[triangle]
\setbeamercolor{item}{fg=gray, bg=white}

\AtBeginSection{}

\usepackage{tabularx}
\usepackage{changepage}
\usepackage{amsmath}
\usepackage{mathrsfs}
\usepackage{mathtools}
\usepackage{xcolor}
\usepackage{adjustbox,lipsum}
\usepackage{graphicx}
\usepackage{booktabs}
\usepackage{multirow}
\usepackage{appendixnumberbeamer} 



\ifLuaTeX
  \usepackage{selnolig}  % disable illegal ligatures
\fi
\usepackage{bookmark}
\IfFileExists{xurl.sty}{\usepackage{xurl}}{} % add URL line breaks if available
\urlstyle{same}
\hypersetup{
  hidelinks,
  pdfcreator={LaTeX via pandoc}}

\title{Final Exam}
\usepackage{etoolbox}
\makeatletter
\providecommand{\subtitle}[1]{% add subtitle to \maketitle
  \apptocmd{\@title}{\par {\large #1 \par}}{}{}
}
\makeatother
\subtitle{Geospatial data analysis in R\\
KDI School Fall 2024}
\author{}
\date{\vspace{-2.5em}2024-11-29}

\begin{document}
\maketitle

\textbf{Due date: Thursday, December 5th at 11:59pm}

Please work by yourself. As before, please submit the following files on eKDIS:

\begin{itemize}
\tightlist
\item
  Your R Markdown file
\item
  Your knitted PDF file
\item
  Any other scripts you used to complete the assignment
\end{itemize}

I would like all of your answers to be in a single markdown file. However, you are welcome to use another script for any of the analyses, if you would prefer. If you do, please include the script in your submission.

\section{Files}\label{files}

Here is the list of files in the folder:

\begin{itemize}
\tightlist
\item
  \texttt{gambiarastershapefile.shp}: This is a shapefile of a grid of Gambia. Each grid is made up of hexagons that are 0.025 degrees from edge to edge.
\item
  \texttt{gambiaroads.shp}: This is a shapefile of primary roads throughout the country.
\item
  \texttt{gmbpop.tif}: This is a raster of population in Gambia.
\item
  \texttt{households.csv}: This has the location of a random sample of households.
\item
  \texttt{prices.csv}: This has the prices of maize at different markets throughout the country.
\end{itemize}

\section{Tasks}\label{tasks}

\begin{enumerate}
\def\labelenumi{\arabic{enumi}.}
\tightlist
\item
  First, create a map -- a single map -- with:
\end{enumerate}

\begin{itemize}
\tightlist
\item
  The grid of Gambia
\item
  The roads
\item
  The location of households
\end{itemize}

\begin{enumerate}
\def\labelenumi{\arabic{enumi}.}
\setcounter{enumi}{1}
\item
  Next, please extract the population values from the raster to the grid. Please create a map of grid cells with fill colors based on the population values.
\item
  Now, please find the distance from each household to the nearest road. Create a histogram of these distances.
\item
  Load the price data.
\end{enumerate}

\begin{itemize}
\tightlist
\item
  Create Voronoi polygons for the markets throughout the country.
\item
  Create a map of the Voronoi polygons with the \emph{household} locations (not market locations).
\end{itemize}

\begin{enumerate}
\def\labelenumi{\arabic{enumi}.}
\setcounter{enumi}{4}
\tightlist
\item
  The final task is the toughest.
\end{enumerate}

\begin{itemize}
\tightlist
\item
  We want to create a raster for maize prices in the country.
\item
  Please use inverse distance weighting to create a raster. Create a (square) raster with a resolution of 0.025 degrees.
\item
  Plot this final raster with a color scale of your choice.
\item
  Find the average maize price in each polygon (from \texttt{gambiarastershapefile.shp}) using the raster you created. Create a map of these average prices.
\end{itemize}

\end{document}
